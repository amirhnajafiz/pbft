\documentclass{article}

% Packages for formatting
\usepackage{geometry}
\usepackage{setspace}
\usepackage{hyperref}
\hypersetup{
    colorlinks=true,
    linkcolor=blue,
    urlcolor=blue,
    pdfborderstyle={/S/U/W 1} % Underline without border
}

% Set margins
\geometry{a4paper}

% Title
\title{Web Scraping from Wikipedia pages}
\author{Amirhossein Najafizadeh}
\date{\today}

\begin{document}

\maketitle

\begin{abstract}
This project aims to gather information from Wikipedia pages and analyze it. Specifically, we'll use web scraping to extract data from these pages. Then, we'll employ techniques to determine what is the page talking about based on its content.
To put it in simple words, we want to categorise our pages.
\end{abstract}

\section{Introduction}
Our project focuses on extracting and analyzing data from Wikipedia pages using web scraping techniques. Moreover, we want
to categorise our pages based on their content. For example, take a look at \href{https://en.wikipedia.org/wiki/Helium}{this page} in Wikipedia about Helium.

\section{Objectives}
Our main objectives are to:
\begin{itemize}
    \item Gather data from Wikipedia pages.
    \item Use web scraping techniques to extract relevant information.
    \item Apply information retrieval techniques to identify page titles based on their content.
\end{itemize}

\section{Methodology}
We'll begin by selecting Wikipedia pages for analysis. Then, we'll employ web scraping methods to extract data from these pages. Finally, we'll use information retrieval techniques to determine the appropriate titles for the extracted content.

\section{Timeline}
Our project will proceed according to the following timeline:
\begin{itemize}
    \item Month 1: Data gathering and initial web scraping.
    \item Month 2: Refining web scraping techniques and beginning information retrieval analysis.
    \item Month 3: Completing information retrieval analysis and finalizing project report.
\end{itemize}

\section{Expected Outcomes}
We anticipate achieving the following outcomes:
\begin{itemize}
    \item Successful extraction of data from Wikipedia pages.
    \item Accurate determination of page titles based on content analysis.
    \item Contribution to the understanding of web scraping and information retrieval techniques.
\end{itemize}

\section{Conclusion}
In conclusion, this project aims to extract and analyze data from Wikipedia pages using web scraping and information retrieval techniques. By achieving our objectives, we hope to contribute valuable insights to the field of data analysis.

% References (if applicable)
% \bibliographystyle{plain}
% \bibliography{references}

\end{document}
